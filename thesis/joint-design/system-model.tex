As shown in Figure \ref{fig:setup}, we aim to investigate the RIS-aided MIMO radar
and MISO multi-user communications. This system serves $K$ users with a signal antenna, which receives
the signal from a BS equipped with $M$ antennas and a RIS equipped with $N$ reflecting elements.
The uniform linear antennas (ULA) are assumed to be the antenna deployment in this model.
The system also works in the tracking mode as a radar to tracking one target at the azimuth 
angle of $\varphi_m$. The overall power budget of this system is $P$. This system model partly follows the work \cite{xu2020tradeoff}.

\subsection{Separated Deployment}

In separated deployment, there are two groups of antennas: $M_r$ radar antennas and $M_c$ communication antennas, which transmit 
radar and communication signals separately. The power budgets of the radar and communications antennas are
$P_r$ and $P_c$, respectively.

The observation at user $k$ is given as
\begin{equation}
    x_k = (\mathbf{h}_k^H \mathbf{\Theta}^H \mathbf{H}_c + {\bf d}_{c,k}^H) \sum_{j=1}^K \mathbf{p}_j s_j + (\mathbf{h}_k^H \mathbf{\Theta}^H \mathbf{H}_r + {\bf d}_{r,k}^H) \mathbf{q} + n_k \label{eq1}
\end{equation}
where $\mathbf{h}_k \in \mathbb{C}^{N \times 1}$ is the channel from
RIS to user $k$, $\mathbf{H}_c \in \mathbb{C}^{N \times M_c}$ and $\mathbf{H}_r \in \mathbb{C}^{N \times M_r}$
are the channels from communication and radar antennas to RIS, respectively. The direct channels from communication 
and radar antennas to user $k$ are denoted by ${\bf d}_{c,k}$ and ${\bf d}_{r,k}$. $s_j$ is the 
information symbol for user $j$, and $n_k \sim \mathcal{CN}(0,\sigma_n^2)$ is the complex 
Gaussian noise at user $k$. The linear precoding
is exploited at the BS and $\mathbf{p}_j \in \mathbb{C}^{M_c \times 1}$ is the linear 
precoder for user $j$. $\mathbf{q} \in \mathbb{C}^{M_r \times 1}$ is the radar signals whose 
covariance matrix is ${\bf{R}}_{\bf q} = \mathbb{E}({\bf{q}}{\bf{q}}^H)$. 
$\mathbf{\Theta} \in \mathbb{C}^{N \times N}$ is the (reflecting) passive beamforming matrix at RIS,
which is modeled as group connected reconfigurable impedance network \cite{shen2020modeling}: 
\begin{align}
    {\bf \Theta} = \mathrm{diag}({\bf \Theta}_1,{\bf \Theta}_2,...,{\bf \Theta}_G) 
    \\ {\bf \Theta}_g = {\bf \Theta}_g^T, {\bf \Theta}_g^H{\bf \Theta}_g = {\bf I}, \forall g
\end{align}
where $G$ is the number of groups in the impedance network and ${\bf \Theta}_g \in \mathbb{C}^{N_G \times N_G}$ is a full matrix. 
$N_G = N / G$ is the number of elements in each group. If $G=1$, the RIS is a fully 
connected network; if $G=N$, the RIS is a single connected network.

The SINR and achievable rate at user $k$ are
\begin{align}
    &\gamma_k=\frac{|{\bf c}_k^H {\bf p}_k|^2}{\sum_{j=1,j \neq k}^K |{\bf c}_k^H {\bf p}_j|^2 + {\bf r}_k^H {\bf R}_{\bf q} {\bf r}_k + \sigma^2_n}\\
    &R_k = \log_2 ( 1 + \gamma_k )
\end{align}
where ${\bf c}_k = \mathbf{H}_c^H \mathbf{\Theta} \mathbf{h}_k+ {\bf d}_{c,k}$
and ${\bf r}_k= \mathbf{H}_r^H \mathbf{\Theta} \mathbf{h}_k + {\bf d}_{r,k}$.

We aim to maximize the WSR as well as the probing power in
direction $\varphi_m$. The WSR is given as
\begin{equation}
    R = \sum_{k=1}^K \mu_k R_k \label{eq5}
\end{equation}
where $\mu_k$ is the weight specified for user $k$ and is determined according the fairness and
quality of service (QoS) requirements.
The probing power in direction $\varphi_m$ is given as
\begin{equation}
    d(\varphi_m) = {\bf{a}}^H(\varphi_m) {\bf{C}} {\bf{a}}(\varphi_m) \label{eq6}
\end{equation}
where ${\bf a}(\varphi_m) \in \mathbb{C}^{M \times 1}$ is the steering 
vector. For ULA deployment, the steering vector is defined as
\begin{equation}
    {\bf{a}}(\varphi_m) = [1, e^{j\frac{2\pi}{\lambda}d\sin({\varphi_m})},...,e^{j\frac{2\pi}{\lambda}d(M-1)\sin({\varphi_m})}]^T \label{eq7}
\end{equation}
where $\lambda$ is the signal wavelength and $d$ is the antenna spacing. 
Without loss of generality, we set $d = \lambda / 2$. ${\bf{C}} \in \mathbb{C}^{M \times M}$ is the covariance matrix of 
the overall transmit signal. As the communication
and radar signals are uncorrelated, the covariance matrix is given as 
\begin{equation}
    {\bf{C}} = \left[\begin{matrix}
        {\bf{R}}_{\bf q} & \bf{0}\\
        \bf{0} & {\bf{P}}{\bf{P}}^H
    \end{matrix}\right] \label{eq8}
\end{equation}
where ${\bf P} = \begin{bmatrix} {\bf p}_1,...,{\bf p}_k\end{bmatrix} $

Therefore, \eqref{eq6} can be rewritten as
\begin{equation}
    d(\varphi_m) = {\bf{a}}_r^H(\varphi_m) {\bf{R}}_{\bf q} {\bf{a}}_r(\varphi_m) + {\bf{a}}_c^H(\varphi_m) {\bf{P}}{\bf{P}}^H {\bf{a}}_c(\varphi_m) \label{eq9}
\end{equation}
where ${\bf{a}}_r(\varphi_m)$ and ${\bf{a}}_c(\varphi_m)$ denote the steering vectors
of radar and communication antennas, respectively.

\subsection{Shared Deployment}

In shared deployment, all the $M$ antennas transmit communication signal. Therefore, the received signal at user $k$ is
\begin{equation}
    \check{x}_k = ({\bf h}_k^H {\bf \Theta}^H {\bf H} + {\bf d}_k^H) \sum_{j=1}^K \check{\bf p}_j s_j
\end{equation}
where ${\bf h}_k \in \mathbb{C}^{N \times 1}$, ${\bf H} \in \mathbb{C}^{M \times N}$, ${\bf d}_k \in \mathbb{C}^{M \time 1}$, 
${\bf \Theta} \in \mathbb{C}^{N \times N}$, $\check{\bf p}_j \in \mathbb{C}^{M \times 1}$, and $s_j \in \mathbb{C}$. 
The SINR at user $k$ is
\begin{equation}
    \check{\gamma}_k = \frac{|\check{\bf c}_k^H \check{\bf p}_k|^2}{\sum_{j=1,j\neq k}^K |\check{\bf c}_k^H \check{\bf p}_j|^2 + \sigma_n^2}
\end{equation}
where $\check{\bf c}_k = {\bf H}^H {\bf \Theta} {\bf h}_k + {\bf d}_k$. Thus, the WSR is
\begin{equation}
    \check{R} = \sum_{k=1}^K \mu_k \check{R}_k
\end{equation}
where $\check{R}_k = \log_2(1+\check{\gamma}_k)$ is the achievable rate at user $k$.

The probing power in direction of $\varphi_m$ is
\begin{equation}
    \check{d}(\varphi_m) = {\bf a}^H(\varphi_m) \sum_{j=1}^K {\bf p}_j {\bf p}_j^H {\bf a}(\varphi_m)
\end{equation}