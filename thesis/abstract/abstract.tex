\begin{abstract}
  The development of the radar system, which has been widely used since the last century, is hindered by interference from
  the next-generation wireless communication system due to the spectrum scarcity problem. Dual-functional radar and 
  communication (DFRC) system, where the radar and communication systems are jointly deployed in a single hardware platform and working 
  simultaneously, is recognized as a promising solution to this problem. However, the achievable region of 
  radar and communication performance is limited by the capability of base station (BS). Thus, in this project,
  we proposed the deployment of reconfigurable intelligent surface (RIS) in DFRC system to enlarge the achievable region.
  The RIS is capable of intelligently controlling the propagation environment of electromagnetic wave by adjusting the 
  magnitude and phase of its reflecting elements.

  Specifically, in this project, we investigated the joint optimization of passive beamforming at RIS and active beamforming
  at a dual-functional BS with separated or shared deployment. In order to maximize the Weighted Sum Rate (WSR) at communication
  users and probing power at target simultaneously, we proposed two alternating optimization (AO) methods based on Weighted
  Minimum Mean Square Error (WMMSE) framework and Fractional Programming (FP) for separated and shared deployments. Unlike 
  many works of literature considering single connected RIS model, a generalized group or fully connected RIS model is exploited in this project.
  Simulation results demonstrate that RIS can help to achieve better transmission beampattern, improve WSR upper bound, and
  enlarge the achievable region.

\end{abstract}
