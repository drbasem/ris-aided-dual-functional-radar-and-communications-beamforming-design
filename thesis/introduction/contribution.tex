In this project, we propose an RIS-aided DFRC active and passive beamforming desgin, which enable communicating 
with the downlink users and probing target in an interesting direction in both separated and shared deployments. 
The following is a list of the project's major contributions.

\begin{enumerate}
    \item We propose a joint beamforming design at BS and RIS that maximizes WSR and the probing power at target for both separated and shared deployment.
    % To the best of our knowledge, this is the first work that studies WSR maximization for RIS-aided DFRC system.
    To our knowledge, this is the first study of WSR maximising for a RIS-aided DFRC system. 
    As in \cite{xu2020tradeoff}, we also consider probing power as radar metric to make a clear tradeoff comparison. However, the logarithm 
    form and additional radar signal term of WSR, the quadratic form of probing power, and the quadratic equality power constraint make the 
    optimization of joint beamforming rather non-convex and elusive.

    \item This is also the first work that investigates the novel group or fully connected RIS model in RIS-aided DFRC. This novel RIS model is capable of enhancing the SNR performance especially in the Rayleigh fading channel \cite{shen2020modeling}, 
    but its potential benefits for enlarging achievable region of WSR and probing power in DFRC are studied in this project.
    
    \item We propose an AO algorithm for the proposed non-convex design in separated deployment.
    As was proposed in \cite{xu2020tradeoff}, the active beamforming problem is reformulated to 
    a semidefinite programming (SDP) problem using WMMSE framework. The passive beamforming 
    problem is reformulated to an unconstrained problem based on Lagrangian dual transform \cite{shen2018fractional2}, quadratic transform \cite{shen2018fractional},
    and scattering-reactance relationship \cite{shen2020modeling}. Compared with the passive beamforming optimization method in 
    \cite{guo2019WSR}, this method is an extended version that considers an additional radar signal term and generalized RIS model.
    
    \item We propose another AO algorithm to solve the non-convex design for shared deployment. 
    The optimal active beamforming can be obtained by WMMSE framework and semidefinite relaxation (SDR) \cite{luo2010semidefinite}. In contrast to 
    the majorization-minimization (MM) method used in \cite{xu2020tradeoff}, which requires several steps to reach the optimal 
    values, this SDR method only needs one step and therefore has lower complexity.
    The passive beamforming is optimized using the same method as separated deployment.

    % \item The simulation results have verified the effectiveness of the proposed algorithm. For both separated and shared deployments,
    % the introduction of RIS helps the system to achieve better radar beampattern and larger achievable region of WSR and probing power at target.
    % Compared with separated deployment, the shared deployment achieves significantly better tradeoff because the antennas are fully exploited.
    % The fully connected RIS brings more improvements in terms of radar beampattern and achievable region than single connected RIS in 
    % Rayleigh channel, while its performance is the same as single connected RIS in LOS channel.
\end{enumerate}





% \begin{enumerate}
%     \item We propose a technique which is capable of maximizing the Weighted Sum Rate (WSR) of communication and the probing power in 
%     an interesting direction of radar for the separated and shared deployment.
%     Instead of SINR at each user used in \cite{jiang2021dfrc}, we use WSR since it is the most representative metric in communication.
%     We also exploit the probing power to replace the received SINR of radar signal used in \cite{jiang2021dfrc}, which can provide more
%     clear tradeoff comparison between communication and radar performance.
%     \item We leverage a novel group or fully connected RIS model instead of the conventional signal connected model exploited in \cite{guo2019WSR,wu2019IRS,guo2020ris,jiang2021dfrc}.
%     This novel RIS model is capable of enhancing the SNR performance especially in the Rayleigh fading channel \cite{shen2020modeling}, 
%     but more benefits in DFRC system is investigate in this project.
%     \item We propose a alternating optimization algorithm to solve the proposed non-convex design of separated deployment. The active beamforming problem is 
%     reformulated to a semidefinite programming (SDP) problem using Weighted Minimum Mean Square Error (WMMSE) framework. The passive beamforming 
%     problem is reformulated to an unconstrained problem based on Lagrangian dual transform \cite{shen2018fractional2}, quadratic transform \cite{shen2018fractional},
%     and scattering-reactance relationship \cite{shen2020modeling}.
%     \item We propose another alternating optimization algorithm to solve the non-convex desgin of shared deployment. The optimal active beamforming 
%     can be obtained by WMMSE framework and Semidefinite Relaxation (SDR) \cite{luo2010semidefinite}. The passive beamforming is optimized using the same method as 
%     separated deployment.
% \end{enumerate}