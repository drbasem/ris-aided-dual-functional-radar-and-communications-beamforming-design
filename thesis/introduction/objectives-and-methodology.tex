In this project, we aim to design a RIS-aided multi-antenna DFRC system and
jointly design active beamforming at a dual-functional BS and passive beamforming at RIS 
to maximize the WSR at users and probing power at target simultaneously.

As was aforementioned, there are only a few works in the field of RIS-aided DFRC \cite{wang2020ris,jiang2021dfrc,wang2021joint}, and we explore
different communication and radar metrics (i.e., WSR and probing power) in this project. According to the qualitative analysis in
Section \ref{sec:background}, a larger achievable region of DFRC system can be realized with RIS. In the following sections, the quantitative
results will be given. Two deployments of dual-functional BS, separated and shared deployments \cite{liu2018beamforming, xu2020tradeoff}, are both considered.
In both deployments, the joint active and passive beamforming design is converted to an optimization problem and solved in an 
alternative manner. The impact of generalized RIS model \cite{shen2020modeling} over DFRC system is compared with conventional 
single-connected RIS model in terms of WSR and probing power enhancement in both Rayleigh and LOS channels.

